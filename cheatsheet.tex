% !TEX program = xelatex

\documentclass[a4paper,11pt]{article}

\usepackage{geometry}

\usepackage{xeCJK}
% \setCJKmainfont{Source Han Sans}  % 思源黑体
\setCJKmainfont{Source Han Serif} % 思源宋体

\usepackage{fontspec}
\newfontfamily\codefontfamily{Menlo}

\usepackage{listings}
\lstset{basicstyle=\footnotesize\codefontfamily,
    captionpos=b,
    breaklines=true,
    mathescape,
    numbers=none,
    tabsize=4,
    showspaces=false,
    frame=lines,
}
\renewcommand{\lstlistingname}{Snippet}

\usepackage{amsmath, amsbsy, amstext, amscd, amsxtra, amsopn, amssymb}
\everymath{\displaystyle}

\newcommand{\obey}{\,\thicksim\,}
\newcommand{\argmax}{\operatorname{arg\,max}}

\newcommand{\app}{a^{\prime}}
\newcommand{\spp}{s^{\prime}}
\newcommand{\thetam}{\theta^{-}}

\title{Reinforcement Learning Cheatsheet}
\author{Izen}

\begin{document}
\maketitle

\section{Deep Q-networks} % (fold)
\label{sec:deep_q_networks}

\begin{align}
	\ell(\theta) &= \mathbb{E}_{(s,a,r,\spp)\obey \mathcal{U}(\mathcal{D})}\left[ \left(y^{\text{DQN}} - Q(s,a;\theta)\right)^2 \right] \\
	y^{\text{DQN}} &= r + \gamma\,Q(\spp,\underset{\app}{\argmax}~Q(\spp,\app;\thetam);\thetam)
\end{align}

% section deep_q_networks (end)

\section{Double Deep Q-networks} % (fold)
\label{sec:double_deep_q_networks}

\begin{align}
	\ell(\theta) &= \mathbb{E}_{(s,a,r,\spp)\obey \mathcal{P}(\mathcal{D})}\left[ \left(y^{\text{DDQN}} - Q(s,a;\theta)\right)^2 \right] \\
	y^{\text{DDQN}} &= r + \gamma\,Q(\spp,\underset{\app}{\argmax}~Q(\spp,\app;\theta);\thetam)
\end{align}

% section double_deep_q_networks (end)

\section{Dueling Networks} % (fold)
\label{sec:dueling_networks}

\begin{align}
	Q(s,a;\theta,\alpha,\beta) &= V(s;\theta,\beta) + A(s,a;\theta,\alpha) \\
	Q(s,a;\theta,\alpha,\beta) &= V(s;\theta,\gamma) + \left( A(s,a;\theta,\alpha)- \underset{\app\in \mathcal{A}}{\max}~A(s,\app;\theta,\alpha) \right) \\
	Q(s,a;\theta,\alpha,\beta) &= V(s;\theta,\gamma) + \left( A(s,a;\theta,\alpha)- \dfrac{1}{\vert \mathcal{A} \vert}\sum_{\app}A(s,\app;\theta,\alpha) \right)
\end{align}

% section dueling_networks (end)

	
\end{document}